\documentclass[main.tex]{subfiles}
\begin{document}
\begin{enumerate}
% -----------------------------------------------------
% Question 1
% -----------------------------------------------------
\item{\textbf{Signum phase mask.} The imaging system shown in Figure \ref{fig:f1} is illuminated by an on-axis plane wave at wavelength $\lambda=\SI{1}{\mu \meter}$. }


\begin{figure}
\centering\fbox{\includegraphics[height=2.0in]{figures/final/1_a_imaging_system.png}}
\caption{Lenses L1, L2 are identical with sufficiently large aperture and focal length $f=\SI{10}{c\metre}$. The pupil mask M has aperture $a=\SI{1}{c\metre}$. Inside the aperture there is a piece of glass of refractive index $n=1.5$. The glass has been partially etched to form a step of height $s=\SI{1}{\mu \metre}$, and the step is precisely aligned with the optical axis, as shown.}
\label{fig:f1}
\end{figure}

\begin{enumerate}
\item{Sketch the amplitude transfer function (AFT) of this optical system.}

\item{A thin transparency with amplitude transmission function}
$$g_t(x)=\cos \left( 2\pi \frac{x}{\SI{20}{\mu \metre} } \right)$$ 
is placed at the object plane. Sketch the magnitude and phase of the transparency $g_t(x)$.

\item{With the transparency $g_t(x)$ in place at the object plane, the system is illuminated with a plane wave on-axis. What is the optical field at the image plane?}

\item{With the same transparency $g_t(x)$ in place at the object plane, the system is illuminated with a plane wave off-axis, propagating at an angle of +0.1rad with respect to the optical axis. What is the optical field at the image plane?}

\item{Suggest an intuitive description of the system's operation under spatially coherent, on-axis plane wave illumination (as in question c)}.

\end{enumerate}

% -----------------------------------------------------
% Question 2
% -----------------------------------------------------
\item{\textbf{Grating with tilted plane wave illumination} Consider a sinusoidal phase grating of the surface relief type with complex amplitude transmission function defined in Equation \ref{eq:fq21}}.

\begin{equation}\label{eq:fq21}
g_{\mathrm{t}}(x)=\exp \left\{i \frac{m}{2} \sin \left(2 \pi \frac{x}{\Lambda}\right)\right\}
\end{equation}

The grating is placed at the plane $z=0$ and illuminated by an off-axis plane wave defined in Equation \ref{eq:fq22} propagating at angle $\theta << 1$ with respect to the optical axis $z$.

\begin{equation}\label{eq:fq22}
\begin{aligned} 
g_{-}(x, z=0)&=\left.\exp \left\{i 2 \pi \frac{x}{\lambda} \sin \theta+i 2 \pi \frac{z}{\lambda} \cos \theta\right\}\right|_{z=0} \\
&\approx \exp \left\{i 2 \pi \frac{\theta x}{\lambda}\right\}
\end{aligned} 
\end{equation}

\begin{enumerate}
\item{Describe, in as much detail as possible, the Fresnel diffraction pattern $g_+(x,z=0)$.}

\item{Describe, in as much detail as possible, the Fraunhofer diffraction pattern.}

\item{Compare with the on-axis illuminated phase grating that we analyzed in class.}


\end{enumerate}
% -----------------------------------------------------
% Question 3
% -----------------------------------------------------
\item{A spherical wave and a plane wave (same wavelength) are co-propagating on axis in air.}

\begin{enumerate}
\item{Describe the interference pattern observed at $z=100\lambda$ from the origin of spherical wave.}

\item{Describe the interference pattern for an off-axis co-propagating plane wave ($\theta_1 = -30$ degrees and $\theta_2 = +45$ degrees in the xz plane with respect to z).}

\item{Is it possible to generate this kind of wave with a Michelson Interferometer? Please motivate and describe your answer extensively.}

\end{enumerate}
% -----------------------------------------------------
% Question 4
% -----------------------------------------------------
\item{Consider a sinusoidal amplitude grating.}
$$g_t(x)=\frac{1}{2}\left[ 1+m\cos(\frac{2\pi x}{\Lambda}) + \phi) \right]$$
at $z=0$. Illuminated by an off-axis plane wave.
$$g_{-}(x,z=0)=\exp(\frac{2\pi}{\Lambda}i \theta x)$$
\begin{enumerate}

\item{Derive the expression of $g_{\text{+}}(x,z=0)$}

\item{What would be the interference pattern at infinity?}
\end{enumerate}
% -----------------------------------------------------
% Question 5
% -----------------------------------------------------
\item{Derive the interference pattern of a plane wave passing through a double split for $D=10\lambda$. Use Huygens' principle and derive the superposition of the waves at a distance $z=L$.}\\

\end{enumerate}
\end{document}