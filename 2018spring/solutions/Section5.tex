\documentclass[main.tex]{subfiles}
\begin{document}
\begin{enumerate}

\subsection*{Section 5} 

\item [13.] In the following:

    \begin{enumerate}
        \item \textbf{Q.} Find the Fourier transform of 
        $$x(t)=\frac{\sin (\pi 2 B t)}{\pi 2 B t} \cos \left(2 \pi f_{c} t\right)$$
        where $f_{c}>2 B>0$. \textbf{Theory.} The Fourier transform (FT) is a transform that converts a function into a form that describes the frequencies present in the original function. The output of the transform is a complex-valued function of frequency. 

        $$
        \hat{x}(\omega)=\int_{-\infty}^{\infty} x(t) e^{-i 2 \pi f t} dt
        $$

        Here, the transform of function $x(t)$ at frequency $\omega$ is denoted by the complex number $\hat{x}(\omega)$, which is just one of several common conventions. All values of $\omega$ produces the frequency domain function. When the independent variable $t$ represents time, the transform variable $\omega$ represents angular frequency. For example, if time is measured in seconds, then frequency is in hertz. Angular frequency is related to frequency by $\omega=2 \pi \xi$. Euler's identity for any real number $x$: $e^{i x}=\cos x+i \sin x$. Euler's identity relationship to trigonometry: $\cos x=\operatorname{Re}\left(e^{i x}\right)=\frac{e^{i x}+e^{-i x}}{2}$, and $\sin x=\operatorname{Im}\left(e^{i x}\right)=\frac{e^{i x}-e^{-i x}}{2 i}$. Complex exponential Fourier transform: 

        $$
        \begin{aligned}
        \mathcal{F}_t [e^{i 2 \pi a t}](f) &= \int_{-\infty}^{\infty} e^{i 2 \pi a t} e^{-i 2 \pi f t} dt \\
        &= \delta(f-a)=\delta(a-f)
        \end{aligned}
        $$

        Sine Fourier transform:

        $$
        \begin{aligned}
        \mathcal{F}_t [\sin (2 \pi A t)](f) &= \int_{-\infty}^{\infty} \frac{e^{i 2 \pi A t}-e^{-i 2 \pi A t}}{2i} e^{-i 2 \pi f t} d t \\
        & =\frac{1}{2i}\left[\int_{-\infty}^{\infty} e^{i 2 \pi A t} e^{-i 2 \pi f t} d t-\int_{-\infty}^{\infty} e^{-i 2 \pi A t} e^{-i 2 \pi f t} dt\right] \\
        &=\frac{1}{2 i}[\delta(f-A)-\delta(f+A)]
        \end{aligned}
        $$

        Cosine Fourier transform:

        $$
        \begin{aligned}
        \mathcal{F}_t [\cos (2 \pi A t)](f) &= \int_{-\infty}^{\infty} \frac{e^{i 2 \pi A t}+e^{-i 2 \pi A t}}{2} e^{-i 2 \pi f t} d t \\
        & =\frac{1}{2}\left[\int_{-\infty}^{\infty} e^{i 2 \pi A t} e^{-i 2 \pi f t} d t+\int_{-\infty}^{\infty} e^{-i 2 \pi A t} e^{-i 2 \pi f t} d t\right] \\
        & =\frac{1}{2}[\delta(f-A)+\delta(f+A)]
        \end{aligned}
        $$
        
        \textbf{A.} Signal $x(t)$ is a converging signal because when $t = \infty$, $x(t) = \frac{1}{\infty} = 0 $ and is thus integrable.

        $$
        \begin{aligned}
        & \mathcal{F}_t\left[\frac{\sin (2 \pi B t) \cos (2 \pi f_c t)}{2 \pi B t}\right](f) \\ 
        &= \frac{1}{2 \pi B} \int_{-\infty}^{\infty} \frac{1}{t} \frac{e^{i 2 \pi B t}-e^{-i 2 \pi B t}}{2i} \frac{e^{i 2 \pi f_c t}+e^{-i 2 \pi f_c t}}{2} e^{-i 2 \pi f t} d t \\
        & = ... \\
        &= \frac{1}{4 \pi B} \left[ -\frac{1}{2} i\left(i \sqrt{\frac{\pi}{2}} \operatorname{sgn}(2 \pi B+\omega-2 f \pi)
        -i \sqrt{\frac{\pi}{2}} \operatorname{sgn}(-2 \pi B+\omega-2 f \pi)\right) \\
        &- \frac{1}{2} i\left( i \sqrt{\frac{\pi}{2}} \operatorname{sgn}(2 \pi B+\omega+2 f \pi)
        -i \sqrt{\frac{\pi}{2}} \operatorname{sgn}(-2 \pi B+\omega+2 f \pi) \right) \right]\\
        &= \frac{\operatorname{sgn}(B-f_c-f)+\operatorname{sgn}(B+f_c-f)+\operatorname{sgn}(B-f_c+f)+\operatorname{sgn}(B+f_c+f)}{8 B}
        \end{aligned}
        $$

        for $B \in \mathbb{R} \wedge f \in \mathbb{R}$.

        \item \textbf{Q.} Find the Hilbert transform $\hat{x}(t) \text { of } x(t)$. \textbf{Theory} \textbf{A.}
        \item \textbf{Q.} Find the analytic signal $\psi(t) \text { of } x(t)$. \textbf{Theory} \textbf{A.}
        \item \textbf{Q.} Find the complex envelope $\gamma(t) \text { of } x(t)$. \textbf{Theory} \textbf{A.}
    \end{enumerate}
    
\item [14.] Assume that $x[n]$ is a real-valued discrete-time signal and $h[n]$ is a real-valued impulse response of linear time-invariant discrete-time system. Let $y_{1}[n]=x[n] \star h[n]$ represent filtering the signal in the forward direction, where $\star$ stands for convolution. Now filter $y_{1}[n]$ backward to obtain $y_{2}[n]=y_{1}[-n] \star h[n]$. The output is then given by reversing $y_{2}[n]$ to obtain $y[n]=y_{2}[-n]$.

    \begin{enumerate}
        \item Show that this set of operation is equivalently represented by a filter with impulse response $h_{o}[n]$ as $y[n]=x[n] \star h_{o}[n]$ and express $h_{o}[n]$ in terms of $h[n]$.
        \item Show that $h_{o}[n]$ is an even signal and find the phase response of a system having impulse response $h_{o}[n]$. Is the system causal?
        \item Let $H(z)$ and $H_{o}(z)$ be z-transforms of $h[n]$ and $h_{o}[n]$, respectively, and that $h[n]$ is causal. If $H(z)=1 /\left(1-0.9 z^{-1}\right)$ find $H_{o}(z)$, the region of convergence of $H_{o}(z)$, and $h_{o}[n]$.
        \item Repeat (c) if $H(z)=1-0.9 z^{-1}$.
    \end{enumerate}
    
\item [15.] Solve the differential equation
    $$y^{\prime \prime}(t)+2 y^{\prime}(t)+y(t)=u(t-1)$$
for $t \geq 0$ using the Laplace transform. $u(t)$ is the unit step function and the initial conditions are $y\left(0^{-}\right)=y^{\prime}\left(0^{-}\right)=1$.

\end{enumerate}
\end{document}