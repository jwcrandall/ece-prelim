\documentclass[main.tex]{subfiles}
\begin{document}
\begin{enumerate}

\subsection*{Section 6: Electronics, Photonics \& MEMS}

\item [16.] Provide clear explanations to the following questions:

    \begin{enumerate}
        \item \textbf{Q.} Using band-theory and energy-related arguments, explain why a metal conducts, an insulator blocks current, and a semiconductor conducts current only under certain situations. \textbf{A.} In metals, the valence band and conduction band overlap, allowing electrons to move freely through the material and conduct electricity. In insulators, the valence band is completely filled with electrons and the conduction band is empty or separated by a large energy gap. This makes it difficult for electrons to move from the valence band to the conduction band and thus blocks current. In semiconductors, the valence band is completely filled with electrons while the conduction band is empty or separated by a small energy gap. This allows electrons to move from the valence band to the conduction band when excited by thermal energy or other means, thus allowing current to flow through the material. Depending on the material and the degree of detail desired, a variety of energy levels will be plotted against position:
        
        \begin{itemize}
            \item $E_{\mathrm{F}}$ or $\mu$ : Although it is not a band quantity, the Fermi level (total chemical potential of electrons) is a crucial level in the band diagram. The Fermi level is set by the device's electrodes. For a device at equilibrium, the Fermi level is a constant and thus will be shown in the band diagram as a flat line. Out of equilibrium (e.g., when voltage differences are applied), the Fermi level will not be flat. Furthermore, in semiconductors out of equilibrium it may be necessary to indicate multiple quasi-Fermi levels for different energy bands, whereas in an out-of-equilibrium insulator or vacuum it may not be possible to give a quasi-equilibrium description, and no Fermi level can be defined.
            \item $E_{\mathrm{C}}$ : The conduction band edge should be indicated in situations where electrons might be transported at the bottom of the conduction band, such as in an $n$-type semiconductor. The conduction band edge may also be indicated in an insulator, simply to demonstrate band bending effects.
            \item $E_{\mathrm{V}}$ : The valence band edge likewise should be indicated in situations where electrons (or holes) are transported through the top of the valence band such as in a p-type semiconductor.
            \item $E_{\mathrm{i}}$ : The intrinsic Fermi level may be included in a semiconductor, to show where the Fermi level would have to be for the material to be neutrally doped (i.e., an equal number of mobile electrons and holes).
            \item $E_{\text {imp }}$ : Impurity energy level. Many defects and dopants add states inside the band gap of a semiconductor or insulator. It can be useful to plot their energy level to see whether they are ionized or not. ${ }^{[3]}$
            \item $E_{\mathrm{vac}}$ : In a vacuum, the vacuum level shows the energy $-e \phi$, where $\phi$ is the electrostatic potential. The vacuum can be considered as a sort of insulator, with $E_{\mathrm{vac}}$ playing the role of the conduction band edge. At a vacuum-material interface, the vacuum energy level is fixed by the sum of work function and Fermi level of the material.
            \item Electron affinity level: Occasionally, a "vacuum level" is plotted even inside materials, at a fixed height above the conduction band, determined by the electron affinity. This "vacuum level" does not correspond to any actual energy band and is poorly defined (electron affinity strictly speaking is a surface, not bulk, property); however, it may be a helpful guide in the use of approximations such as Anderson's rule or the Schottky-Mott rule.
        \end{itemize}

        \begin{figure}
        \centering\fbox{\includegraphics[width=3.0in]{2018spring/figures/16a_a.png}}
        \caption{Band theory of insulator, semiconductor, and metal}
        \label{fig:17q_a}
        \end{figure}
        
        \item \textbf{Q.} A PN junction is used as a photodetector. Assume that light is shining on all parts of the diode equally. Which part of the photodiode is most critical for photo detection and why? In this application, should the devices be under forward or reverse bias? 
        
        \textbf{A.} Forward biasing means putting a voltage across a diode that allows current to flow easily, while reverse biasing means putting a voltage across a diode in the opposite direction. A photodiode is a semiconductor device with a P-N junction that converts photons (or light) into electrical current. The P layer has an abundance of holes (positive), and the N layer has an abundance of electrons (negative). Photodiodes can be manufactured from a variety of materials including, but not limited to, Silicon, Germanium, and Indium Gallium Arsenide. Each material uses different properties for cost benefits, increased sensitivity, wavelength range, low noise levels, or even response speed. 

        \begin{figure}
        \centering\fbox{\includegraphics[width=4.0in]{2018spring/figures/16a_b.png}}
        \caption{P-N Photodiode Cross-section}
        \label{fig:16a_b}
        \end{figure}
        
        Figure \ref{fig:16a_b} shows a cross section of a typical photodiode. A \textbf{Depletion Region} is formed from diffusion of electrons from the N layer to the P layer and the diffusion of holes from the P layer to the N layer. This creates a region between the two layers where no free carriers exist. This develops a built-in voltage to create an electric field across the depletion region. This allows for current to flow only in one direction (Anode to Cathode). The photodiode can be forward biased, but current generated will flow in the opposite direction. This is why most photodiodes are reversed biased or not biased at all. Some photodiodes cannot be forward biased without damage.

        A photon can strike an atom within the device and release an electron if the photon has enough energy. This creates an electron-hole pair (e- and h+) where a hole is simply an “empty space” for an electron. If photons are absorbed in either the P or N layers, the electron hole pairs will be recombined in the materials as heat if they are far enough away (at least one diffusion length) from the depletion region. Photons absorbed in the depletion region (or close to it) will create electron hole pairs that will move to opposite ends due to the electric field. Electrons will move toward the positive potential on the Cathode, and the holes will move toward the negative potential on the Anode. These moving charge carriers form the current (photocurrent) in the photodiode. Figure \ref{fig:16a_b} shows the different layers of a photodiode (P-N Junction) as well as multiple connection points on top and bottom. The depletion region creates a capacitance in the photodiode where the boundaries of the region act as the plates of a parallel plate capacitor. Capacitance is inversely proportional to the width of the depletion region. Reverse bias voltage also influences the capacitance of the region. There are four major parameters used in choosing the right photodiode and whether or not to reverse bias the photodiode.

        \begin{itemize}
            \item Response (speed/time) of the photodiode is determined by the capacitance of the P-N junction. It is the time needed for charge carriers to cross the P-N junction. This is directly affected by the width of the depletion region.
            \item Responsivity is the ratio of photocurrent generated from incident light, to that incident light power. This is usually expressed in units of A/W (current over power). A typical responsivity curve of a photodiode will show A/W as a function of wavelength. This is called Quantum Efficiency.
            \item Dark current is the current in the photodiode when there is no incident light. This can be one of the main sources of noise in the photodiode system. Photocurrent from background radiation can also be included in this measurement. Photodiodes are usually put into an enclosure that does not allow any light to hit the photodiode to measure the dark current. Because the current generated by the photodiode can be very small, dark current levels can obscure the current produced by incident light at low light levels. Dark current increases with temperature. Without biasing, the dark current can be very low. The ideal photodiode would have no dark current.
            \item Breakdown Voltage is the largest reverse voltage that can be applied to the photodiode before there is an exponential increase in leakage current or dark current. Photodiodes should be operated below this maximum applied reverse bias or damage to the photodiode may occur. Breakdown voltage decreases with an increase in temperature.
        \end{itemize}

        \begin{figure}
        \centering\fbox{\includegraphics[width=3.0in]{2018spring/figures/16a_c.png}}
        \caption{PHOTOCONDUCTIVE MODE REVERSE BIASED}
        \label{fig:16a_c}
        \end{figure}

        When the photodiode is reverse biased (Figure \ref{fig:16a_c}), an external voltage is applied to the P-N junction. The negative terminal is connected to the positive P layer, and the positive terminal is connected to the negative N layer. This causes the free electrons in the N layer to pull toward the positive terminal, and the holes in the P layer to pull toward the negative terminal. When the external voltage is applied to the photodiode, the free electrons start at the negative terminal and immediately fill the holes in the P layer with electrons. This creates negative ions in the atoms with extra electrons. The charged atoms then oppose the flow of free electrons to the P layer. Similarly, holes go about the same process to create positive ions but in the opposite direction. When reverse biased, current will only flow through the photodiode with incident light creating photocurrent. The reverse bias causes the potential across the depletion region to increase and the width of the depletion region to increase. This is ideal for creating a large area to absorb the maximum amount of photons. The response time is reduced by the reverse bias by increasing the size of the depletion layer. This increased width reduces the junction capacity and increases the drift velocity of the carriers in the photodiode. The transit time of the carriers is reduced, improving the response time. Unfortunately, increasing the bias current increases the dark current as well. This noise can be a problem for very sensitive systems using P-N or PIN photodiodes. This hinders the performance in low light situations. If using APDs, the signal to noise ratio will be large regardless because of the gain of the photodiode. Because a photon is ideally absorbed in the depletion region, the P layer can be constructed to be extremely thin. This can be balanced with the reverse bias to create an optimal photodiode with a faster response time while maintaining as low as noise as possible.
        
        \item \textbf{Q.} After repeated operations of a PMOS MOSFET, hole-type interface traps are formed in the Si-SiO2 interface, does this process increase or decrease the threshold voltage? Draw the band-diagram and the sub-threshold IV curve to illustrate your answer. \textbf{A.}
        
    \end{enumerate}

\item [17.] MOS Capacitor

    \begin{enumerate}
        \item Draw the band diagram of a MOS system where the "metal" work function $\Phi_{\mathrm{M}}$ is larger than the silicon work function $\Phi_{\mathrm{S}}$. Assume that there are no applied voltages at the p-type substrate (doping $\mathrm{N}_{\mathrm{A}}$) and the gate. On the diagram clearly label the following parameters and functions: electron affinity in the semiconductor $\chi_{\mathrm{Sc}}$, the Fermi level $\mathrm{E}_{\mathrm{F}}$, the conduction and valance band edges $\mathrm{E}_{\mathrm{c}}$ and $\mathrm{E}_{\mathrm{V}}$, the band-gap $\mathrm{E}_{\mathrm{g}}$, the mid-gap $\mathrm{E}_{i}$, the thickness of the oxide $\mathrm{t}_{\mathrm{ox}}$, the potential drop in the oxide $\phi_{\mathrm{ox}}$, and the potential drop in the semiconductor $\phi(\mathrm{x})$
        \item For this device, what is the most likely outcome when no voltages are applied: inversion or accumulation? Why?
        \item Poly-Silicon Gate Depletion (refer to Figure \ref{fig:17q_a}): Assume the voltage $\mathrm{V}_{\mathrm{ox}} = \qty{1}{\volt}$ across a $\qty{2}{\nano\meter}$ thin \ch{SiO2} oxide. The $\mathrm{P}^{+}$ poly gate doping is $\mathrm{N}_{\text {poly }}=1 \times 10^{19} \mathrm{~cm}^{-3}$ and the substrate is n-doped with $\mathrm{N}_{\mathrm{D}}=10^{17} \mathrm{~cm}^{-3}$. Find the poly depletion width, $\mathrm{W}_{\mathrm{dep }}$.
    \end{enumerate}
    
\begin{figure}
\centering\fbox{\includegraphics[width=3.0in]{figures/2018s/17q_a.png}}
\caption{Schematic of the poly depletion capacitances upon gating this MOS capacitor. $\mathrm{T}=300\mathrm{K}$}. Gate and body are Silicon. The gate ocide is \ch{SiO2}, $\mathrm{t}_{\mathrm{ox}} = 2 \mathrm{~nm}$}
\label{fig:17q_a}
\end{figure}

\item [18.] Basic \textit{pn}-junction operation.\\

Consider the ideal so-called "long-base" abrupt \textit{pn}-junction silicon diode that has a uniform cross section and constant doping on both sides of the \textit{pn}-junction. The diode is doped as follows: $\mathrm{N}_{\mathrm{a}}=8.0 \times 10^{16} \mathrm{~cm}^{-3}$ \textit{p}-type and $\mathrm{N}_{\mathrm{d}}=1 \mathrm{x} 10^{16} \mathrm{~cm}^{-3}$ \textit{n}-type. For this material, the minority-carrier lifetimes are: $\tau_{\mathrm{n}}=4 \times 10^{-6} \mathrm{~s}$ and $\tau_{\mathrm{p}}=1 \times 10^{-6} \mathrm{~s}$, respectively. You may assume that the effects within the space-charge region are negligible and that the minority carriers flow only by diffusion in the charge neutral regions.

    \begin{enumerate}
        \item Draw/sketch the band-diagram for this system. Also, plot the electrostatic potential, the net charge density and the and the corresponding electric field.
        \item Determine the value of the built-in potential across the \textit{pn}-junction.
        \item Calculate the density of the minority carriers at the edge of the space-charge region for a forward bias of 0.3V.
        \item Under bias condition, calculate and plot the minority and majority carrier currents as a function of distance from the junction.
    \end{enumerate}

\end{enumerate}
\end{document}